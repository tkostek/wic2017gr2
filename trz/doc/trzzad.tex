\documentclass[zad,zawodnik,utf8]{sinol}
\usepackage{graphicx}
    \title{Trzy Źródła}
    \id{trz}
    \signature{tkos-kas}
    \author{Tomek Kościuszko}

    \pagestyle{fancy}
    \iomode{stdin}
    \konkurs{\small{Warsztaty Informatyczne 2017 - grupa 2}}

%     \day{Dzień 123}
    \date{25.12.2222}

    \RAM{256}

\begin{document}
\begin{tasktext}%
    Rok 2000 BC, wódz OG III stanął przed problemem przeludnienia w swojej wiosce. Dostępne źródła pożywienia wyczerpują się i trzeba znaleźć nowe miejsce, gdzie będzie można wybudować kolonię i przenieść tam część mieszkańców wioski. Wraz ze swoją radą plemienną wódz wybrał teren w kształcie kwadratu o boku 10km, na którym znajdują się trzy wystarczające źródła pożywienia. Miejsce jest dobre do kolonizacji o ile znajduje się nie dalej niż $R$ kilometrów od najbliższego źródła, w obrębie wyznaczonego terenu, tzn obie współrzędnę są nieujemne i nie większe niż 10.

    Znając lokalizaję trzech źródeł w układzie współrzednych, wyznacz pole figury złożonej ze wszystkich punktów na mapie zdatnych do zamieszkania.

\section{Wejście}
    W pierwszym wierszu podana jest liczba całkowita $0 \leq R \leq 10$ - pożądana odległość od najbliższego punktu zaopatrzenia.
    W następnych trzech wierszach podane są współrzędne każdego ze źródeł: $0 \leq x_{i}, y_{i} \leq 10$.

\section{Wyjście}
    Należy podać szukane pole z dokładnością do $0.1$ km$^2$, wolno się pomylić maksymalnie o $0.1$ km$^2$.


\makeexample{0a}

\makeexample{0b}


\end{tasktext}
\end{document}
