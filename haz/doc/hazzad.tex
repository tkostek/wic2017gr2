\documentclass[zad,zawodnik,utf8]{sinol}
\usepackage{graphicx}
    \title{Hazard}
    \id{haz}
    \signature{tkos-kas}
    \author{Tomek Kościuszko}

    \pagestyle{fancy}
    \iomode{stdin}
    \konkurs{\small{Warsztaty Informatyczne 2017 - grupa 2}}

%     \day{Dzień 123}
    \date{25.12.2222}

    \RAM{256}

\begin{document}
\begin{tasktext}%
    Gdy o późnej godzinie wszystkie grzeczne dzieci już śpią, w zaułkach Bajtockich ulic gromadzą się czarne charaktery. Bywają noce, kiedy z przyjemnością uprawiają hazard. Od niedawna popularnością cieszy się nowa gra - ,,Parka''. Proste zasady i dynamiczna rozgrywka sprawiają, że łatwo wyzbyć się całej gotówki, jaką się posiada.

    W ,,Parce'' rzuca się jedną, $n$ - ścienną kostką. Za każdy rzut klient płaci określoną kwotę. Jeśli w dwóch kolejnych rzutach wypadnie jedynka, gra się kończy i wypłacana jest wcześniej ustalona nagroda. Wydawałoby się, że łatwo zarobić trochę kieszonkowego wchodząc do gry, często okazuje się jednak, że ,,Parka'' nie pojawia się wyjątkowo długo...

    Tom Sromotnik - za dnia spokojny golibroda, w noc przemienia się w miastowego potentata gier hazardowych. Pomóż mu wyznaczyć jaka jest oczekiwana liczba rzutów\footnote{Powiedzmy, że rozegraliśmy $k$ gier, które skończyły się odpowiednio po $a_{1}, a_{2}, ..., a_{k}$ rzutach. Weźmy średnią arytmetyczną tych liczb. Tym większą będziemy wybierać liczbę $k$, tym średnia będzie bliższa oczekiwanej liczbie rzutów} $n$ - ścienną kostką, po których wypada ,,Parka'', tzn. dwie jedynki pod rząd.

\section{Wejście}
    Na wejściu podana jest jedynie jedna liczba całkowita: $3 \leq n \leq 100$.

\section{Wyjście}
    Oczekiwana liczba rzutów, dopuszczalny błąd wynosi $10^{-6}$.


\makeexample{0}


\end{tasktext}
\end{document}
